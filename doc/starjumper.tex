\documentstyle{llncs}

\begin{document}

\title{Starjumper}
\author{Robert Strobl, Johannes Albrecht, Tim Berning, Stefan H\"artel}
\institute{Hasso-Plattner-Institut, Prof.-Dr.-Helmert-Str. 2-3, 14482 Potsdam}

\maketitle

\section{Einleitung}

\subsection{Spielprinzip}
Starjumper ist ein Renn- und Geschicklichkeitsspiel, bei dem der Spieler mit einem
Raumschiff \"uber eine hindernisreiche Strecke mit diversen Abgr\"unden fahren und
schlussendlich sicher ins Ziel gelangen muss.\\
Dabei besitzen die Fl\"achen der Streckenabschnitte unterschiedliche Eigenschaften,
die sich auf das Fahrverhalten des Raumschiffs auswirken k\"onnen, z.B. Abbremsen
des Spielers.\\
Die Steuerung umfasst Beschleunigen, Bremsen, Lenken nach links oder rechts und
Springen.

\subsection{Spielidee}
Vorlage f\"ur unser Spiel war der Spieleklassiker Skyroads aus dem Jahr 1993, welches
von Bluemoon entwickelt wurde und wiederum selbst eine Neuauflage des Spiels Kosmonaut
aus dem eigenen Hause darstellte.\\
Das Spielprinzip wurde f\"ur Starjumper gr\"o\ss tenteils \"ubernommen.



\section{Architektur}

Im Folgenden sollen nun die wichtigsten architektonischen Bestandteile von Starjumper
erl\"autert werden.

\subsection{GameManager}
\subsection{Menu}
\subsection{Game}
\subsection{Level}
\subsection{Player}



\section{Designentscheidungen}



\section{Design Patterns ololol}



\section{Diskussion und Ausblick}



\end{document}